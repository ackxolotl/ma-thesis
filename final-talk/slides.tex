\documentclass[NET,english,beameralt]{tumbeamer}

% If you load additional packages, do so in packages.sty as figures are build
% as standalone documents and you may want to have effect on them, too.

% Folder structure:
% .
% ├── beamermods.sty                  % depricated an will be removed soon
% ├── compile                         % remotely compile slides
% ├── figures                         % all figures go here
% │   └── schichtenmodelle_osi.tikz   % each .tikz or .tex is a target
% ├── include                         % create your document here
% │   ├── example.tex                 % example document
% │   └── slides.tex                  % make document wide changes here
% ├── lit.bib                         % literature
% ├── Makefile
% ├── moeptikz.sty                    % fancy networking symbols
% ├── packages.sty                    % load additional packages there
% ├── pics                            % binary pcitures go here
% ├── slides.tex                      % main document (may be more than one)
% ├── tumbeamer.cls
% ├── tumcolor.sty                    % TUM color definitions
% ├── tumcontact.sty                  % TUM headers and footers
% ├── tumlang.sty                     % TUM names and language settings
% └── tumlogo.sty                     % TUM logos

% Configure author, title, etc. here:
\usepackage[utf8]{inputenc}
\usepackage{packages}
\usepackage{beamermods}

% For beamer mode (default):
\author[S. Ellmann]{Simon Ellmann}
\title[Effects of hardware isolation]{Investigating effects of hardware isolation in high-speed network environments}

% Uncomment to add advisors for presentation in Oberseminar
\advisor{Paul Emmerich, Benedikt Jaeger, Florian Wiedner}

% Uncomment to add type for presentation in Oberseminar
% Usage: \thesistype{intermediate | final}{bachelor | master | idp | gr}
%\thesistype{intermediate}{gr}

% Uncomment to configure date in dd-mm-yyyy format
\newdate{date}{12}{05}{2021}
\date{\displaydate{date}}

% Uncomment to add a venue to the title slide
%\venue{International Conference on Conferencing 2018 \\ Garching b. München, Germany}

% For lecture mode (use package option 'lecture'):
%\lecture[GRNVS]{Grundlagen Rechnernetze und Verteilte Systeme}
%\module{IN0010}
%\semester{SoSe\,2016}
%\assistants{Johannes Naab, Stephan Günther, Maurice Leclaire}


\usepackage{pgfpages}
\usepackage{ifthen}
% ============================================================================
% jobname solution
% ============================================================================
\newif\ifsolution%
\ifthenelse{\equal{\detokenize{notes}}{\jobname}}{%
\setbeameroption{show notes on second screen=bottom}
\setbeamercolor{note page}{bg=white, fg=black}
\setbeamercolor{note title}{bg=white!95!black, fg=black}
}{
}

% TeXLive 2018 compatibility: https://tex.stackexchange.com/questions/426088/texlive-pretest-2018-beamer-and-subfig-collide
\makeatletter
\let\@@magyar@captionfix\relax
\makeatother


\begin{document}

% If you are preparing a talk but do not like the default font sizes, you may
% want to try the class option 'beameralt', which uses smaller default font
% sizes and integrates subsection/subsubsection names into the headline.

% For lecture mode, you may want to build one set of slides per chapter but
% with common page numbering. If so,
% 1) create a new .tex file for each chapter, e.g. slides_chapN.tex,
% 2) set the part counter to N-1 (assuming chapters start at 0), and
% 3) and name your chapter by using the \part{} command.
%\setcounter{part}{-1}
%\part{Organisatorisches und Einleitung}

% For 16:9 slides, use the class option 'aspectratio=169'.

% If class option 'noframenumbers' is given, frame numbers are not printed.

% If class option 'notitleframe' is given, the title frame is not autmatically
% generated.

% Class option 'nocontentframes' suppresses automatic generation of content
% frames when new parts/sections are started.

% Include source files from ./include (or ./include/chapN).
\chapter{Introduction}
\label{chap:introduction}

TODO



% Include markdown source from ./pandoc
%\input{pandoc/example}

% Comment out if you do not want a bibliography
%\section{Bibliography}
%\begin{frame}[allowframebreaks]
%    \bibliographystyle{abbrv}
%    \setbeamertemplate{bibliography item}[text]
%    \footnotesize
%    \bibliography{lit}
%\end{frame}

\end{document}

