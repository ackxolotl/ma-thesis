\chapter{Implementations}
\label{chap:implementations}

To analyse performance, safety and security implications of \acp{iommu}, we use
\texttt{ixy.rs} \cite{ellmann2018writing}, a high-speed user space network
driver written in Rust for Intel 10G \acp{nic} of the \texttt{ixgbe} family, and
adapt it to our needs. Changes to the existing implementation and new code are
described in the following sections.

Since \texttt{ixy.rs} is a network driver written for Linux on x86, all remarks
on specific behavior refer to this combination of operating system and
architecture.


\section{ixy.rs}
\label{sec:ixy_rs}

Stefan Huber added basic \ac{iommu} support to \texttt{ixy.rs} in 2019 through
Linux's \ac{vfio} API \cite{huber2019using}. We complete the implementation by
adding support for the following two use cases: accessing multiple devices
associated to the same \ac{iommu} group and using \texttt{ixy.rs} in conjunction
with legacy \acp{iommu} not supporting 48-bit wide \acp{iova}.

For the former, only minor changes in the code are necessary. In the original
implementation, a device's \ac{vfio} group file is opened on device setup and
closed after the \ac{vfio} device file descriptor was derived from the group.
However, a device group file cannot be re-opened if it is used through a derived
device file descriptor, even if the descriptor of the group file was properly
closed.  Re-opening the group file in that case returns OS error 16, i.e.,
``device or resource busy''. We mitigate this problem by keeping \ac{vfio} group
files opened and storing the file descriptors in a \texttt{HashMap<i32, RawFd>}
that maps \ac{iommu} groups to group file descriptors. When a new device is
initialized, the \texttt{HashMap} is checked for the \ac{vfio} group. In case
there is no key-value pair for the group, the group file is opened and group
number and descriptor are added to the \texttt{HashMap}. Otherwise, the group
file descriptor from the \texttt{HashMap} is used.

To support \acp{iommu} with an \ac{iova} widths smaller than the host's virtual
address widths (e.g., only 39 instead of 48 bits), smaller \acp{iova} have to be
chosen. Currently, \texttt{ixy.rs} uses a 1:1 mapping of process virtual
addresses to \acp{iova} when setting up the \acp{dmar}. Thus, \ac{mmu} and
\ac{iommu} map the same (\ac{io}) virtual addresses to the same physical
addresses and there is no need for \texttt{ixy.rs} to track which \ac{iova} is
used for which \ac{dma} buffer. We keep this mapping for simplicity but enforce
all memory to be used by the \ac{nic} to be mapped into the lower 32-bit of the
process address space. To our knowledge, there are no \acp{iommu} for the x86
architecture that do not support at least 32-bit addresses.

Mapping memory into the 32-bit address space of a process is no witchcraft.
\texttt{mmap} can be forced through the \texttt{MAP\_32BIT} flag to create
mappings in the first 4 GiB of the process address space (indeed, only 2 GiB are
used, making some authors call it the \texttt{MAP\_31BIT} flag). However,
mapping memory gets more sophisticated when huge pages are used (as they are in
\texttt{ixy.rs}) due to alignment requirements: huge pages have to be
huge-page-aligned, i.e., at 2 MiB or 1 GiB boundaries.

To obey these requirements, we use \texttt{mmap} twice: to determine an
appropriate address range and to map the huge pages into the address range.
\Cref{lst:32-bit-map} illustrates the algorithm. First, a mapping of needed
size + one huge page size is created (whereby needed size is equivalent to one
or multiple huge pages). We know that any address range of size $s$ contains a
$s$-aligned address, i.e., for any \texttt{mmap}-returned address $r$ we know
that ${\exists a \in \{r, r + 1, ..., r + s - 1\}: a \equiv 0 \mod s}$. Thus we
use the additionally mapped huge page to find the huge-page-aligned address $a$.
After that, excess memory is freed, i.e., memory between the
\texttt{mmap}-returned address and the huge-page-aligned address and any
remaining bytes of the additionaly mapped huge page at the end of the mapping.
Finally, huge pages of the needed size are mapped to the huge-page-aligned
address by a second call to \texttt{mmap}, passing the aligned address as first
argument to \texttt{mmap} and the flag \texttt{MAP\_FIXED} to force
\texttt{mmap} to map the huge pages at the specified address.

\begin{minipage}{\textwidth}
    \lstinputlisting[label={lst:32-bit-map},language=Rust,caption={Mapping huge
        pages into the first 2 GiB of the process address space.},captionpos=b]
        {code/32-bit-map.rs}
\end{minipage}

Besides these newly implemented features in \texttt{ixy.rs}, some minor changes
were made to the source code. We fixed a bug in the memory module where virtual
addresses where used instead of \ac{io} virtual addresses, a wrong bitmask was
discovered and corrected in the original ixy driver \cite{emmerich2019user} that
propagated into all re-implementations \cite{emmerich2019case} of the driver,
and some comments were updated, typos fixed and overall code cleanliness
improved (removal of superfluous casts, parantheses, etc.).


\section{ixgbevf}
\label{sec:ixgbevf}

To analyse the effects of \acp{iommu} in context of \ac{sriov}, we implement
\texttt{ixgbevf} for \texttt{ixy.rs}, a variant of the \texttt{ixgbe} driver for
virtual functions (VFs).

While \texttt{ixgbe} (physical function driver) and \texttt{ixgbevf} (virtual
function driver) have great similarities, device setup and communication between
driver and \ac{nic} differ, and there are some tasks on virtual functions (like
reset) that have to be executed cooperatively by both drivers. Besides these
differences in the drivers, there are also differences in the devices: the PCI
configuration space of virtual functions returns \texttt{0xffff} for vendor and
device ID, making it necessary to grab the values from files in the
\texttt{sysfs} or from the physical functions' configuration spaces to determine
the correct driver to be used, and virtual functions support only up to 8 RX/TX
queues in contrast to 64 of the physical function.

Reset and initialization of the physical function, i.e., the device itself, and
creation of virtual functions is initiated by the physical function driver. The
physical function driver brings the \ac{nic} into an operational state and
creates virtual functions when requested to do so, e.g., by the \ac{os}.

Reset and initialization of the virtual function is initiated by the virtual
function driver. To reset the VF, communication with the physical function
driver is necessary. In general, VF driver communication happens in two
different ways: Directly with the \ac{nic} via memory-mapped \ac{io} and by
passing messages to the physical function driver via a mailbox system. For
direct communication with the \ac{nic}, only a part of the device configuration
space can be accessed by the virtual function. The registers in this
configuration space allow the VF driver to set up the descriptor rings, enable
or disable interrupts, send and receive packets, etc. For operations that have
global impact like configuring VLAN filters or resetting the virtual function,
the VF driver has to communicate with the PF driver and request the needed
operations. This is done via the mailbox system that is implemented in hardware
in the \ac{nic}.

Messages used for mailbox communication are hardware-independent and can be
freely chosen by driver implementations. However, to be able to use our driver
in conjunction with the kernel \texttt{ixgbe} driver as PF driver, messages must
be coordinated and synchronized. Thus, we implement the same mailbox messages as
the Linux kernel driver and DPDK.

\Cref{lst:mbx} shows how communication with the \ac{nic} mailbox is
implemented in \texttt{ixgbevf}. \linebreak \texttt{read\_msg\_from\_mbx} is
used to read a message from the mailbox buffer. It is expected to be called once
a new message for the virtual function arrives (which can be determined through
the \texttt{IXGBE\_VFMAILBOX} register). The method takes a slice \texttt{msg}
of unknown size to store the message from the mailbox. Before the message is
copied to the slice, the mailbox is locked to prevent race conditions between VF
and PF. Subsequently, the message is copied and receipt of the message
acknowledged (releasing the mailbox at the same time).

\begin{minipage}{\textwidth}
    \lstinputlisting[label={lst:mbx},language=Rust,caption={Reading a message
    from the VF mailbox.},captionpos=b] {code/mbx.rs}
\end{minipage}

Virtual function initialization of \texttt{ixgbevf} resembles to a large extent
physical function initialization of \texttt{ixgbe}: Once the VF's registers are
mapped into memory and DMA bus master is enabled for the virtual function,
interrupts are disabled and the VF is reset by prompting the physical function
driver for function reset. When the PF driver has acknowledged function reset,
the MAC address of the virtual function is derived from the PF through mailbox
communication or generated locally. Subsequently, RX and TX descriptor rings are
set up and initialization of the virtual function is complete.

Unlike with \texttt{ixgbe}, there is no link setup since the physical function
driver is responsible for the link. Indeed, virtual functions can only be used
once the device's link is up. If the link is down, virtual function reset hangs.

Deriving the MAC address and using it as source address for outgoing packets is
crucial for \texttt{ixgbevf} as the kernel driver enables MAC anti-spoofing by
default, i.e., all TX packets are checked for deviating MAC addresses by the
\ac{nic} and discarded if necessary.


\section{iommu-leaks}
\label{sec:iommu_leaks}

For precise time measurements on \ac{nic} operations and to analyse effects of
the \ac{iotlb}, we implement some additional functionality in \texttt{ixy.rs}.

To be independent of the physical link we add methods to enable loopback
operations on the \ac{nic}. In loopback mode, packets are transmitted from the
\ac{nic}'s TX queues via its internal 10 Gigabit Media Independent Interface
(XGMII) to the RX queues without leaving the device.

We also add methods to dynamically enable and disable queues. This is
advantageous to approximate how much time is needed for \ac{dma} transfers. By
disabling all RX queues, TX transmit times can be broken down to a \ac{dma}
access to the TX queue page and another \ac{dma} access to the packet buffer.

To measure \ac{cpu} cycles for \ac{nic} operations, we use the x86
\texttt{rdtsc} instruction. We implement a rdtsc function as shown in
\Cref{lst:rdtsc}. We use intrinsic functions to avoid inline assembler. Since
intrinsic functions are inherently unsafe, we need a \texttt{unsafe} block.
Note our use of memory fences to serialize instruction execution.
\texttt{lfence}s deter the compiler from reordering code and the \ac{cpu} from
speculatively executing subsequent instructions, namely the \texttt{rdtsc}
itself and following instructions. Unlike suggested by a Intel in a benchmarking
guide for the IA-32 and IA-64 architecture \cite{paoloni2010benchmark}, we use
\texttt{lfence} instead of \texttt{cpuid} as it provides the same serialization
guarantees and keeps the variance small.

\begin{minipage}{\textwidth}
    \lstinputlisting[label={lst:rdtsc},language=Rust,caption={\texttt{rdtsc()}
        function to count \acs{cpu} cycles.},captionpos=b]
        {code/rdtsc.rs}
\end{minipage}

To be able to evaluate measured \ac{cpu} cycles, we implement a logger that logs
values to a file-backed huge page and provides statistics about the measured
values like mean, variance and sample variance.

For \ac{iotlb}-related research, we implement a brute-force memory allocator
that allocates physically and virtually contiguous 4 KiB pages. By using normal
sized pages, locality of packet buffers can be maintained while the amount of
pages to be translated by the \ac{iommu} respectively the \ac{iotlb} can be
controlled variably.

The memory allocator works as follows: First, two memory regions are allocated
via \texttt{mmap}, a huge region -- which is our page pool -- containing 1,024
pages that are checked for contiguity and a target memory area. Next, virtual
and physical address of every page in the page pool are inserted into some data
structure, e.g., a vector, and sorted by ascending physical addresses.
Subsequently, the sorted pages are remapped to the target memory area using
\texttt{mremap} and checked for contiguity. Since all pages have been remapped
in physically contiguous order, physically adjacent pages are now also adjacent
in virtual address space. Thus, it suffices to iterate through the target memory
area and check whether there is a block of physically subsequent pages big
enough to satisfy the requested amount of contiguous memory.

To control memory placement of RX/TX queues and packet buffers, we add methods
to \texttt{ixy.rs} that reinitialize the aforementioned data structures at
variable memory addresses, e.g., memory allocated by our brute-force memory
allocator. We also add methods for fine-grained control of TX descriptors and
queues that allow us to repeatedly use the same descriptor(s) for transmit.

