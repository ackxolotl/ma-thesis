\chapter{Implementations}
\label{chap:implementations}

To analyse performance, safety and security implications of \acp{iommu}, we use
\texttt{ixy.rs} \cite{ellmann2018writing}, a high-speed user space network
driver written in Rust for Intel 10G \acp{nic} of the \texttt{ixgbe} family, and
adapt it to our needs. Changes to the existing implementation and new code are
described in the following sections. All architectural and \ac{os}-related
remarks refer to Linux on x86-64, the target platform of \texttt{ixy.rs}.


\section{ixy.rs}
\label{sec:ixy_rs}

Stefan Huber added basic \ac{iommu} support to \texttt{ixy.rs} in 2019 through
Linux's \ac{vfio} API \cite{huber2019using}. We complete the implementation by
adding support for the following two use cases: accessing multiple devices
associated to the same \ac{iommu} group and using \texttt{ixy.rs} in conjunction
with legacy \acp{iommu} not supporting 48-bit wide \acp{iova} for second-level
address translation (i.e., the translation mode used by \ac{vfio}).

For the former, only minor changes in the code are necessary. In the original
implementation, a device's \ac{vfio} group file is opened on device setup and
closed after the \ac{vfio} device file descriptor was derived from the group.
However, a device group file cannot be re-opened if it is used through a derived
device file descriptor, even if the descriptor of the group file was properly
closed. Re-opening the group file in that case returns \ac{os} error 16, i.e.,
``device or resource busy''. We mitigate this problem by keeping \ac{vfio} group
files opened and storing the file descriptors in a \texttt{HashMap<i32, RawFd>}
that maps \ac{iommu} groups to group file descriptors. When a new device is
initialized, the \texttt{HashMap} is checked for the \ac{vfio} group. In case
there is no key-value pair for the group, the group file is opened and group
number and descriptor are added to the \texttt{HashMap}. Otherwise, the group
file descriptor from the \texttt{HashMap} is used.

To use \texttt{ixy.rs} with \acp{iommu} with a maximum second-level translation
\ac{iova} widths smaller than the host's virtual address widths (e.g., only 39
instead of \SI{48}{\bit}), we reduce the \ac{iova} widths in \texttt{ixy.rs}.
Currently, \texttt{ixy.rs} uses a 1:1 mapping of process virtual addresses to
\acp{iova} when setting up the \acp{dmar}. Hence, \ac{mmu} and \ac{iommu} map
the same (\ac{io}) virtual addresses to the same physical addresses and there is
no need for \texttt{ixy.rs} to track which \ac{iova} is used for which \ac{dma}
buffer. We keep this mapping for simplicity but enforce all memory to be used by
the \ac{nic} to be mapped into the lower \SI{32}{\bit} of the process address
space. To our knowledge, there are no \acp{iommu} for the x86 architecture that
do not support addresses up to \SI{39}{\bit}.

On a side note: a 1:1 mapping of process virtual addresses to \acp{iova} is one
of the use cases mentioned in Intel's \ac{iommu} specification as Shared Virtual
Memory (SVM) \cite[p.~21]{intel2019iommu}. Unfortunately, we cannot make use of
SVM since neither the \texttt{ixgbe} \acp{nic} nor Linux's \ac{vfio} framework
support the required \ac{pcie} feature (PASID). In general, PASID seems to be a
rather rare feature: Rothwell states in his dissertation that he has not seen it
used in practice \cite[p.~31]{rothwell2018exploitation}.

Mapping memory into the 32-bit address space of a process is not rocket science.
\texttt{mmap} can be forced through the \texttt{MAP\_32BIT} flag to create
mappings in the first \SI{4}{\gibi\byte} of the process address space (indeed,
only \SI{2}{\gibi\byte} are used, which is why some authors call it the
\texttt{MAP\_31BIT} flag). However, mapping memory gets more sophisticated when
huge pages are used (as they are in \texttt{ixy.rs}) due to alignment
requirements: huge pages have to be huge-page-aligned, i.e., at
\SI{2}{\mebi\byte} or \SI{1}{\gibi\byte} boundaries. Due to the alignment
requirements, calling \texttt{mmap} with \texttt{MAP\_32BIT} and
\texttt{MAP\_HUGETLB} fails.

To obey the alignment, we use \texttt{mmap} twice: to determine an appropriate
address range and to map the huge pages into the address range.
\Cref{lst:32-bit-map} illustrates the algorithm. First, a mapping of needed size
plus one huge page size is created (where needed size is equivalent to one or
multiple huge pages). We know that any address range of size $s$ contains a
$s$-aligned address, i.e., for any address $r$ returned by \texttt{mmap} we know
that ${\exists a \in \{r, r + 1, ..., r + s - 1\}: a \equiv 0 \mod s}$. Thus we
use the additionally mapped huge page to find the huge-page-aligned address $a$.
After that, excess memory is freed, i.e., memory between the \texttt{mmap}-
returned address and the huge-page-aligned address and any remaining bytes of
the additionaly mapped huge page at the end of the mapping. Finally, huge pages
of the needed size are mapped to the huge-page-aligned address by a second call
to \texttt{mmap}, passing the aligned address as first argument to \texttt{mmap}
and the flag \texttt{MAP\_FIXED} to force \texttt{mmap} to map the huge pages at
the specified address.

\begin{minipage}{\textwidth}
    \lstinputlisting[label={lst:32-bit-map},language=Rust,caption={Mapping huge
    pages into the first \SI{2}{\gibi\byte} of the process address space.},
    captionpos=b]{code/32-bit-map.rs}
\end{minipage}

Besides these newly implemented features in \texttt{ixy.rs}, some minor changes
were made to the source code. We fixed a bug in the memory module where virtual
addresses where used instead of \ac{io} virtual addresses, a wrong bitmask was
discovered and corrected in the original ixy driver \cite{emmerich2019user} that
propagated into all re-implementations \cite{emmerich2019case} of the driver,
and some comments were updated, typos fixed and overall code cleanliness
improved (removal of superfluous casts, parentheses, etc.).


\section{ixgbevf}
\label{sec:ixgbevf}

To analyse the effects of \acp{iommu} in context of \ac{sriov}, we implement
\texttt{ixgbevf} for \texttt{ixy.rs}, a variant of the \texttt{ixgbe} driver for
\ac{sriov} \acp{vf}. We base our implementation on the \texttt{ixgbevf} code in
the Linux kernel and DPDK, and the Intel 82599 SR-IOV Companion Guide
\cite{intel201082599}.

While \texttt{ixgbe} (\ac{pf} driver) and \texttt{ixgbevf} (\ac{vf} driver)
share large parts of their code base, device setup and communication between
driver and \ac{nic} differ, and there are some tasks on \acp{vf} (like reset)
that have to be executed cooperatively by both drivers. Besides these
differences in the drivers, there are also differences in the devices: \acp{vf}
support only up to 8 RX/TX queues in contrast to 64 of the \ac{pf}, and the PCI
configuration space of \acp{vf} returns \texttt{0xffff} for vendor and device
ID, making it necessary to grab the values from files in the \texttt{sysfs} or
from the \acp{pf}' PCI configuration spaces to determine the correct driver to
be used.

Reset and initialization of the \ac{pf}, i.e., the device itself, and creation
of \acp{vf} is initiated by the \ac{pf} driver. The \ac{pf} driver brings the
\ac{nic} into an operational state and creates \acp{vf} when requested to do so,
e.g., by the \ac{os}.

Reset and initialization of the \ac{vf} is initiated by the \ac{vf} driver. To
reset the \ac{vf}, communication with the \ac{pf} driver is necessary. In
general, \ac{vf} driver communication happens in two ways: Directly with the
\ac{nic} via memory-mapped \ac{io} and by passing messages to the \ac{pf} driver
via a mailbox system. For direct communication with the \ac{nic}, only a part of
the device configuration space can be accessed by the \ac{vf}.  The registers in
this configuration space allow the \ac{vf} driver to set up the descriptor
rings, enable or disable interrupts, send and receive packets, etc.  For
operations that have global impact like configuring VLAN filters or resetting
the \ac{vf}, the \ac{vf} driver has to communicate with the \ac{pf} driver and
request the needed operations. This is done via a mailbox system that is
implemented in hardware in the \ac{nic}.

Messages used for mailbox communication are hardware-independent and can be
freely chosen by driver implementations. However, to be able to use our driver
in conjunction with the kernel \texttt{ixgbe} driver as \ac{pf} driver, we have
to use the same messages for the same operations. Hence, we implement the
mailbox in the way the Linux kernel driver and DPDK do.

\Cref{lst:mbx} shows how communication with the \ac{nic} mailbox is implemented
in \texttt{ixgbevf}. \linebreak \texttt{read\_msg\_from\_mbx} is used to read a
message from the mailbox buffer. It is expected to be called once a new message
for the \ac{vf} arrives (which can be determined through the
\texttt{IXGBE\_VFMAILBOX} register). The method takes a slice \texttt{msg} of
unknown size to store the message from the mailbox. Before the message is copied
to the slice, the mailbox is locked to prevent race conditions between \ac{vf}
and \ac{pf}. Subsequently, the message is copied and receipt of the message
acknowledged (releasing the mailbox at the same time).

\begin{minipage}{\textwidth}
    \lstinputlisting[label={lst:mbx},language=Rust,caption={Reading a message
    from the VF mailbox.},captionpos=b] {code/mbx.rs}
\end{minipage}

\ac{vf} initialization of \texttt{ixgbevf} resembles to a large extent \ac{pf}
initialization of \texttt{ixgbe}: Once the \ac{vf}'s registers are mapped into
memory and \ac{dma} bus master is enabled for the \ac{vf}, interrupts are
disabled and the \ac{vf} is reset by prompting the \ac{pf} driver for function
reset. When the \ac{pf} driver has acknowledged function reset, the \ac{mac}
address of the \ac{vf} is derived from the \ac{pf} through mailbox communication
or generated locally. Subsequently, RX and TX descriptor rings are set up and
initialization of the \ac{vf} is complete.

Unlike in \texttt{ixgbe}, there is no link setup since the \ac{pf} driver is
responsible for the link. Indeed, \acp{vf} can only be used once the device's
link is up. If the link is down, \ac{vf} reset stalls. If the link goes down
while \acp{vf} are used, the \acp{vf} become inoperable and have to be reset
once the \ac{pf}'s link is up again.

Deriving the \ac{mac} address and using it as source address for outgoing
packets is crucial for \texttt{ixgbevf} as the kernel driver enables \ac{mac}
anti-spoofing by default, i.e., all TX packets are checked for deviating
\ac{mac} addresses by the \ac{nic} and get discarded if a deviating address was
detected.


\section{iommu-leaks}
\label{sec:iommu_leaks}

For precise time measurements on \ac{nic} operations and to analyse effects of
the \ac{iotlb}, we implement some additional functionality in \texttt{ixy.rs}.

To be independent of the physical link, we add methods to enable loopback
operations on the \ac{nic}. In loopback mode, packets are transmitted from the
\ac{nic}'s TX queues via its internal 10 Gigabit Media Independent Interface
(XGMII) to the RX queues without leaving the device.

We also add methods to dynamically enable and disable queues. This is
advantageous to approximate how much time is needed for \ac{dma} transfers. By
disabling all RX queues, TX transmit times can be broken down to a \ac{dma}
access to the TX queue page and another \ac{dma} access to the packet buffer.

To measure \ac{cpu} cycles for \ac{nic} operations, we use the x86
\texttt{rdtsc} instruction. We implement a rdtsc function as shown in
\Cref{lst:rdtsc}. We use intrinsic functions to avoid inline assembler. Since
intrinsic functions are inherently unsafe, we need an \texttt{unsafe} block.
Note our use of memory fences to serialize instruction execution.
\texttt{lfence}s deter the compiler from reordering code and the \ac{cpu} from
speculatively executing subsequent instructions, namely the \texttt{rdtsc}
itself and following instructions. Unlike suggested by a Intel in a benchmarking
guide for the IA-32 and IA-64 architecture \cite{paoloni2010benchmark}, we use
\texttt{lfence} instead of \texttt{cpuid} as it provides the same serialization
guarantees and keeps the variance small.

\begin{minipage}{\textwidth}
    \lstinputlisting[label={lst:rdtsc},language=Rust,caption={\texttt{rdtsc()}
        function to count \acs{cpu} cycles.},captionpos=b]
        {code/rdtsc.rs}
\end{minipage}

To be able to evaluate measured \ac{cpu} cycles, we implement a logger that logs
values to a file-backed huge page and provides statistics about the measured
values like mean, variance and sample variance.

For \ac{iotlb}-related research, we implement a brute-force memory allocator
that allocates physically and virtually contiguous 4 KiB pages. By using normal
sized pages, locality of packet buffers can be maintained while the amount of
pages to be translated by the \ac{iommu} respectively the \ac{iotlb} can be
controlled variably.

The memory allocator works as follows: First, two memory regions are allocated
via \texttt{mmap}, a huge region -- which is our page pool -- containing 1,024
pages that are checked for contiguity and a target memory area. Next, virtual
and physical address of every page in the page pool are inserted into some data
structure, e.g., a vector, and sorted by ascending physical addresses.
Subsequently, the sorted pages are remapped to the target memory area using
\texttt{mremap} and checked for contiguity. Since all pages have been remapped
in physically contiguous order, physically adjacent pages are now also adjacent
in virtual address space. Thus, it suffices to iterate through the target memory
area and check whether there is a block of physically subsequent pages big
enough to satisfy the requested amount of contiguous memory.

To control memory placement of RX/TX queues and packet buffers, we add methods
to \texttt{ixy.rs} that reinitialize the aforementioned data structures at
variable memory addresses, e.g., memory allocated by our brute-force memory
allocator. We also add methods for fine-grained control of TX descriptors and
queues that allow us to repeatedly use the same descriptor(s) for transmit.

