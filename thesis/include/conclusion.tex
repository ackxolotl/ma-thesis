\chapter{Conclusion}
\label{chap:conclusion}

We investigated the effects of hardware isolation in high-speed network
environments with a focus on Intel's and AMD's \acp{iommu}. We used the
\texttt{ixy.rs} user space network driver for our analyses, and adapted it to
our needs. We implemented support for multiple devices in an \ac{iommu} group
and \acp{iommu} with limited second-level \ac{iova} translation capabilities. We
also implemented a whole new driver in \texttt{ixy.rs} for virtual functions,
\texttt{ixgbevf}. And we implemented methods for precise timing measurements on
\ac{nic} operations to investigate the \ac{iotlb} of \acp{iommu}, including a
brute-force memory allocator to allocate physically contiguous memory in user
space.

Our findings with \texttt{ixy.rs} paint a multi-layered picture of \acp{iommu}.
In regards to performance, our results show that on most configurations, neither
throughput nor latency are affected by \acp{iommu} to a quantifiable extent, be
it in virtualized or non-virtualized environments. There are, however, some
configurations where \acp{iommu} have a negative impact, leading to a loss of
packet throughput of more than 50\%.

In regards to safety and security, our picture of \acp{iommu} is rather bleak:
Against common believe, \acp{iommu} do not protect against malicious or faulty
peripherals due to the weaknesses in the \ac{pcie} protocol and archaic threat
models of \ac{os} vendors which see internal \ac{io} peripherals as trustworthy
and enable inherently unsafe features like ATS on these devices.

As a result, many weaknesses in the handling of \acp{iommu} have been uncovered
in the past, as well as some architectural deficiencies. We believe that more
vulnerabilities are to be discovered and suspect one in the hardware of
\acp{iommu}: the \ac{iotlb}. As other caches, we believe that the \ac{iotlb} is
susceptible to timing attacks and thus provides an attack surface for side
channel attacks. If timing of \ac{dma} accesses and address translation through
the \ac{iotlb} can be measured precisely, packet receival and transmittal times
of other devices sharing the \ac{iotlb} could be leaked.

Our measurements on \ac{dma} operations provide some insight on the performance
characteristics of Intel's \ac{iommu}. We hope these results can be used in the
future to uncover the suspected vulnerability in the \ac{iotlb}.

Generally speaking of safety and security, however, we recognize that a lot has
changed in the last years: More and more \acp{os} support \acp{iommu} or improve
their usage of \acp{iommu} when mitigating vulnerabilities, more vendors of
consumer products include \acp{iommu} into their hardware to protect systems
against external peripherals, and even on mobile phones \acp{iommu} are on the
rise: Apple has developed its own \ac{iommu} for the iPhone, Google uses an
\ac{iommu} on the Pixel phones, and ARM-based Android phones often have an ARM
SMMU included.

In view of these rapid changes, we believe that effects of hardware isolation
remain an exciting research topic in all kinds of environments, not necessarily
restricted to virtualization and high-speed networking.


\section{Future Work}
\label{chap:future_work}

Future work could continue our work on the \ac{iotlb}, investigate \acp{iommu}
on mobile platforms, e.g., the \ac{iommu} of Apple's iPhones, or focus on
Linux's \ac{vfio} API and its different ways of handling \ac{iova} address
mappings, e.g., coalescing of addresses and mapping with different page sizes.

