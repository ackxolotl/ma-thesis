\DeclareAcroListStyle{longtabu}{table}{%
    table = longtabu,
    table-spec = @{}>{}lX@{}
}{%

\acsetup{%
    list-style=longtabu,
    extra-style=plain,    % remove dot after long in list
    only-used=false,
}

\tabulinesep=1ex

\DeclareAcronym{io}{
    short               = {\sc{I/O}},
    long                = {input/output},
    list                = {\Acl{io}.},
    extra               = {%
        Flow of information between an information processing system and a human
        or another information processing system.
    }
}
\DeclareAcronym{cpu}{
    short               = {\sc{CPU}},
    short-plural-form   = {\sc*{CPU}s},
    long                = {central processing unit},
    long-plural-form    = {central processing units},
    list                = {\Acl{cpu}.},
    extra               = {%
        Electronic circuitry that executes arithmetic, logic, control and
        \acs{io} instructions specified by computer programs.
    }
}
\DeclareAcronym{dma}{
    short               = {\sc{DMA}},
    long                = {direct memory access},
    list                = {\Acl{dma}.},
    extra               = {%
        Feature of computer systems that allows hardware to access main memory
        independent of the \acs{cpu}.
    }
}
\DeclareAcronym{sriov}{
    short               = {\sc{SR-IOV}},
    long                = {single root input/output virtualization},
    list                = {\Acl{sriov}.},
    extra               = {%
        Feature of computer systems that allows a single \ac{pcie} device to be
        shared with multiple virtual environments.
    }
}
\DeclareAcronym{mmu}{
    short               = {\sc{MMU}},
    short-plural-form   = {\sc*{MMU}s},
    long                = {memory management unit},
    long-plural-form    = {memory management units},
    list                = {\Acl{mmu}.},
    extra               = {%
        Computer hardware component that primarily translates virtual to
        physical memory addresses. Often part of the \ac{cpu}.
    }
}
\DeclareAcronym{iommu}{
    short               = {\sc{IOMMU}},
    short-plural-form   = {\sc*{IOMMU}s},
    long                = {input-output memory management unit},
    long-plural-form    = {input-output memory management units},
    list                = {\Acl{iommu}.},
    extra               = {%
        A \ac{mmu} for \ac{io} devices.
    }
}
\DeclareAcronym{smmu}{
    short               = {\sc{SMMU}},
    short-plural-form   = {\sc*{SMMU}s},
    long                = {system memory management unit},
    long-plural-form    = {system memory management units},
    list                = {\Acl{smmu}.},
    extra               = {%
        ARM's term for \ac{iommu}.
    }
}
\DeclareAcronym{tlb}{
    short               = {\sc{TLB}},
    short-plural-form   = {\sc*{TLB}s},
    long                = {translation lookaside buffer},
    long-plural-form    = {translation lookaside buffers},
    list                = {\Acl{tlb}.},
    extra               = {%
        Memory cache of a \ac{mmu} that stores recent address translations to
        improve performance.
    }
}
\DeclareAcronym{iotlb}{
    short               = {\sc{IOTLB}},
    short-plural-form   = {\sc*{IOTLB}s},
    long                = {I/O translation lookaside buffer},
    long-plural-form    = {I/O translation lookaside buffers},
    list                = {\Acl{iotlb}.},
    extra               = {%
        \ac{tlb} of an \ac{iommu}.
    }
}
\DeclareAcronym{nic}{
    short               = {\sc{NIC}},
    short-plural-form   = {\sc*{NIC}s},
    long                = {network interface controller},
    long-plural-form    = {network interface controllers},
    list                = {\Acl{nic}.},
    extra               = {%
        Computer hardware component that receives and transmits data via a
        computer network.
    }
}
\DeclareAcronym{os}{
    short               = {\sc{OS}},
    short-plural-form   = {\sc*{OS}es},
    long                = {operating system},
    long-plural-form    = {operating systems},
    list                = {\Acl{os}.},
    extra               = {%
        Low-level software that manages a computer's hardware and software and
        provides services for computer programs.
    }
}
\DeclareAcronym{pcie}{
    short               = {\sc{PCIe}},
    long                = {Peripheral Component Interconnect Express},
    list                = {\Acl{pcie}.},
    extra               = {%
        High-speed serial bus standard designed to replace PCI.
    }
}
\DeclareAcronym{bios}{
    short               = {\sc{BIOS}},
    short-plural-form   = {\sc*{BIOS}es},
    long                = {Basic Input/Output System},
    long-plural-form    = {Basic Input/Output Systems},
    list                = {\Acl{bios}.},
    extra               = {%
        Firmware that initializes a computer's hardware on boot before the
        \ac{os} is loaded.
    }
}
\DeclareAcronym{uefi}{
    short               = {\sc{UEFI}},
    long                = {Unified Extensible Firmware Interface},
    list                = {\Acl{uefi}.},
    extra               = {%
        Standardized successor of the \acp{bios} found in IBM PC-compatible
        computers.
    }
}
\DeclareAcronym{vfio}{
    short               = {\sc{VFIO}},
    long                = {Virtual Function I/O},
    list                = {\Acl{vfio}.},
    extra               = {%
        Linux driver framework that utilizes \acp{iommu} and allows direct
        device access from user space in a safe manner.
    }
}
\DeclareAcronym{iova}{
    short               = {\sc{IOVA}},
    short-plural-form   = {\sc*{IOVA}s},
    long                = {I/O virtual address},
    long-plural-form    = {I/O virtual addresses},
    list                = {\Acl{iova}.},
    extra               = {%
        Address used by \ac{io} devices for \ac{dma} access. \acp{iommu}
        translate IOVAs to physical adresses.
    }
}
\DeclareAcronym{acpi}{
    short               = {\sc{ACPI}},
    long                = {Advanced Configuration and Power Interface},
    list                = {\Acl{acpi}.},
    extra               = {%
        Open standard for discovery and configuration of computer hardware
        components by \acp{os}.
    }
}
\DeclareAcronym{dmar}{
    short               = {\sc{DMAR}},
    long                = {DMA remapping},
    list                = {\Acl{dmar}.},
    extra               = {%
        Component of \acp{iommu} responsible for address translation. Often
        used synonymously with \ac{iommu}.
    }
}
\DeclareAcronym{vf}{
    short               = {\sc{VF}},
    short-plural-form   = {\sc*{VF}s},
    long                = {virtual function},
    long-plural-form    = {virtual functions},
    list                = {\Acl{vf}.},
    extra               = {%
        Lightweight \ac{pcie} function of \ac{sriov}-capable devices that is
        associated with a \ac{pf} and shares resources with its \ac{pf} and
        other \acp{vf}. Often passed-through to virtual environments.
    }
}
\DeclareAcronym{pf}{
    short               = {\sc{PF}},
    short-plural-form   = {\sc*{PF}s},
    long                = {physical function},
    long-plural-form    = {physical functions},
    list                = {\Acl{pf}.},
    extra               = {%
        Fully featured \ac{pcie} function that supports \ac{sriov}. Creates
        and manages \acp{vf} and shared resources such as the link.
    }
}
\DeclareAcronym{bdf}{
    short               = {\sc{BDF}},
    long                = {bus/device/function},
    list                = {\Acl{bdf}.},
    extra               = {%
        Identifier triple of \ac{pcie} devices.
    }
}
\DeclareAcronym{mac}{
    short                = {\sc{MAC}},
    long                = {medium access control},
    list                = {\Acl{mac}.},
    extra               = {%
        Sublayer of the data link layer responsible for flow control and
        multiplexing of the transmission medium.
    }
}
\DeclareAcronym{sdn}{
    short               = {\sc{SDN}},
    long                = {software-defined networking},
    list                = {\Acl{sdn}.},
    extra               = {%
        Approach to dynamic network management by splitting switches into
        remotely configurable control planes (routing information) and data
        planes (packet forwarding).
    }
}
\DeclareAcronym{nfv}{
    short               = {\sc{NFV}},
    long                = {network function virtualization},
    list                = {\Acl{nfv}.},
    extra               = {%
        Creation of building blocks by virtualizing network functionality
        traditionally implemented in hardware (e.g., NAT, firewalls), and
        chaining of virtualized blocks to create communication services.
    }
}
