\small

Hardware-Isolation spielt in heutigen Computersystemen eine wichtige Rolle.
Internetfirmen und Telekommunikationsanbieter sind mit ihrer stark
virtualisierten Infrastruktur auf Hardware-Isolation angewiesen, und
Verbraucherprodukte nutzen sie, um sich vor böswilligen oder fehlerhaften
externen Geräten zu schützen.

Eine gängige Methode zur Isolierung von Hardware ist die Verwendung von IOMMUs,
Mehrzweckgeräten zur Virtualisierung von IO-Speicher, die heutzutage in allen
Arten von Geräten zu finden sind, von High-End-Servern bis hin zu mobilen
Geräten wie Apples iPhones, und die für schnelle IO-Operationen auf Plattformen
wie Amazons Elastic Cloud Computing unumgänglich sind. Aufgrund der weiten
Verbreitung von IOMMUs stellt sich die Frage, inwiefern sie die Leistung und die
Sicherheit der Systeme, in denen sie eingesetzt werden, beeinflussen.

Wir adressieren diese Frage im Bereich von
Hochgeschwindigkeits-Netzwerkumgebungen. Für unsere Analyse verwenden wir
\texttt{ixy.rs}, einen in Rust geschriebenen, modernen
User-Space-Netzwerktreiber für Intels 82599-Netzwerkkarten. Wir zeigen, dass
IOMMUs in den meisten Fällen einen geringen Einfluss auf die Performance haben.
In manchen Fällen jedoch führen IOMMUs zu einem Performanceverlust von mehr als
50\%. Während sich unser Verdacht auf eine Seitenkanal-Schwachstelle in den
Übersetzungspuffern von IOMMUs nicht bestätigt, stellen wir fest, dass der
Schutz von IOMMUs gegenüber bösartigen Geräten gering ist.

Mit dieser Arbeit tragen wir einen neuen Treiber zu \texttt{ixy.rs} für
virtuelle Funktionen, Unterstützung für ältere Intel-IOMMUs mit eingeschränkter
Second-Level-Adressübersetzung und ein Werkzeug zur Durchführung genauer
Zeit-Messungen auf Speicherzugriffe von Netzwerkkarten bei. Des Weiteren wurde
ein Fehler in DPDKs ixgbevf-Treiber behoben.

